\documentclass[a4paper, 12pt]{article}

%% Language and font encodings
\usepackage[T1]{fontenc}
\usepackage[utf8x]{inputenc}
\usepackage{natbib}
\usepackage[french]{babel}

%Demande de l'INSPE de pas d'hyphénation mais rend le texte non justifié...
%\usepackage[none]{hyphenat}

% À decommenter si obligatoire. À vérifier aurpès de Maria...
%\usepackage{helvet}
%\renewcommand{\familydefault}{\sfdefault}
%\usepackage{setspace}
%\onehalfspacing


%\usepackage{multirow}
%\usepackage{tabularx}
%\usepackage{wrapfig} 
%\usepackage{subfig}
%\usepackage{picins}

% Defining colors
\usepackage{xcolor}
\definecolor{deepblue}{rgb}{0.3,0.3,0.8}
\definecolor{darkblue}{rgb}{0,0,0.3}
\definecolor{deepred}{rgb}{0.6,0,0}
%\definecolor{iremred}{RGB}{204,35,50}% ROUGE IREM
%\definecolor{deepgreen}{rgb}{0,0.6,0}
%\definecolor{backcolor}{rgb}{0.98,0.95,0.95}
%
\usepackage{hyperref}
%
%\usepackage{tocloft}
%\usepackage{xparse}
%\usepackage{etoolbox}
%\urlstyle{tt}
\newcommand{\email}[1]{\href{mailto:#1}{\tt{\nolinkurl{#1}}}}

\usepackage[parfill]{parskip}
\usepackage{fancyhdr}
\usepackage{authblk}
\renewcommand\Authand{ et }			%franciser
\renewcommand\Authands{, et }		%franciser
%\usepackage{calc}
%\usepackage{inconsolata}
%\setlength{\headheight}{40pt}


%% Sets page size and margins
%\usepackage[a5paper,top=3cm,bottom=2cm,left=2cm,right=2cm,marginparwidth=1.75cm]{geometry}
%%%%%%écrire dans la marge
\usepackage[fulladjust]{marginnote}
\usepackage[a4paper,margin=2cm]{geometry}% À changer en 2.5 selon les demandes de Maria : obligatoire ?
\reversemarginpar
\newcommand{\idee}[1]{%
\reversemarginpar
\hspace{-0.4em}\marginnote{\footnotesize #1}%
}
\newcommand{\info}[1]{\normalmarginpar\marginpar{
  \textcolor{deepred}{\texttt{
  \begin{minipage}[t]{1.5cm}
    \begin{flushleft}
      \tiny #1
    \end{flushleft}
  \end{minipage}
  }}}}
  

%% Useful packages
%\usepackage{amsmath}

\usepackage{graphicx}
%\usepackage{subfig} 

% renommer les figures
%\addto\captionsfrench{%
%  \renewcommand{\listfigurename}{Liste des captures}%
%  \renewcommand{\figurename}{\textsc{Capture}}%
  %\renewcommand{\listtablename}{Nouveau nom}%
%}

%\usepackage{booktabs}
%\usepackage[colorinlistoftodos]{todonotes}
%---
%\usepackage[most,listings,breakable,theorems]{tcolorbox}  
%%%%%%%%%%%%%%%%%%%%%%%%%%%%%%%%%
% nomenclature

%Mettre dans crochets: "number within=section," <- pour numérotation sous la forme section/soussection du code
% cf doc de tcolorbox ici (469p!) :
%http://texdoc.net/texmf-dist/doc/latex/tcolorbox/tcolorbox.pdf
%\newtcbtheorem[auto counter,list inside=cmb]{CMB}{Code \no}
%{top=-1.5mm,bottom=-2mm,fonttitle=\sffamily\bfseries,arc=0mm, colback=backcolor,colframe=iremred}%
%{code}%<-pour ref{code:...}
%%%
%%%%-----


%\newcommand{\ciitice}{\gls{cii} \gls{tice}}

%%%%-----------------------------
%           TITRE et h/b de page
%%%%-----------------------------
%\renewcommand{\headrulewidth}{0pt}

\title{{\includegraphics[width=\linewidth]{espe}}\\[2cm]
Formation des stagiaires à temps complet\\[2cm]Écrit réflexif\\2021-2022 \\[4cm]
\textbf{Constitution de groupes de niveau à l'aide d'activités en ligne}\\[2cm]
}
\vspace{2cm}

\author{Pascal Padilla -- Lyc\'ee Simone Veil, Marseille \email{pascal.padilla@ax-aix-marseille.fr}}
%\affil[1]{}%
%\affil[2]{}%

\date{}


%%
%\fancyhead[C]{\includegraphics[width=0.5 \linewidth]{espe}}
%\fancyhead[R]{\includegraphics[width=2.5cm]{LogoC2it_800.png}}
%\fancyhead[R]{}
\fancyhead[L]{}

\fancyfoot[L]{Pascal Padilla}
\fancyfoot[R]{INSPE, Formation Stagiaire 2021-22}
\renewcommand{\footrulewidth}{1pt}

\pagestyle{fancy}


\usepackage{lipsum}
\newcommand{\remplissage}{
  {\color{deepblue}\lipsum[1][1-10]}}


% Citations perso
\newcommand{\textcite}[1]{\og \textit{#1} \fg{}}


%%--##################################################
\begin{document}
\renewcommand{\labelitemi}{\textbullet}

%%--##################################################
\maketitle

\thispagestyle{empty}
%Page de titre

\newpage
%%--##################################################
\begin{abstract}

Dans ce document, je vais présenter comment j'ai utiliser un outil informatique pour créer des groupes de niveaux dans ma classe. 

Merci à \href{nicole.mencacci@univ-amu.fr}{Nicole Mencacci} pour sa formation et son accompagnement.

\end{abstract}

\thispagestyle{empty}

%%--##################################################
\newpage
%sommaire
\tableofcontents

\thispagestyle{empty}

%%--##################################################
\newpage
%%--##################################################
%
%  				DEBUT du texte principal
%
%--##################################################

\newgeometry{margin=2cm,left=5cm,marginparwidth=4cm}
\fancyhfoffset[L]{3cm}


\section{Exposé et analyse de la situation professionnelle problématique}

Je suis actuellement enseignant de spécialité Numérique et Sciences Informatique (NSI) en lycée général. J'enseigne au lycée Simone Veil de Marseille (13013), établissement qui accueille 850 élèves.


Sur cet établissement, un élève de première doit choisir trois spécialité parmi une dizaine disponible sur le bassin. La NSI est un des choix possible.

On y apprend l'informatique en tant que science. C'est-à-dire l'architecture et les communications entre les machines, l'algorithmique, la traduction en langage de programmation ou encore la gestion des données.


Durant l'année, j'ai eu à mettre en place une activité s'appuyant sur la technique Dijksow

\section{Données recueillies}

Grâce à l'ENT Moodle, je parviens à collecter un grand nombre de données.

\section{Conclusion}

J'ai réussi à disciminer mes élèves en fonction de leur réussite






\end{document}
