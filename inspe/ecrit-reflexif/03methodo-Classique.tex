
\subsection{L'appropriation \og classique \fg{} des notions : groupes témoins}

Tout d'abord, il convient de définir ce que nous concevons comme étant une appropriation \og classique \fg{} des notions. En effet, même s'il existe beaucoup de différences entre les pratiques pédagogiques, nous constatons qu'un courant pédagogique prédomine en France. Ce courant divise les séances de cours en quatre types : séance de cours, travaux dirigés (TD), travaux pratiques (TP) et évaluation. Bien entendu, au sein d'une même séance, plusieurs types peuvent être pratiqués. Chaque enseignant peut ainsi proposer à ses élèves différentes activités en fonction du type utilisé, mais cette grande catégorisation reste un commun dans l'enseignement français.

Nous nous proposons donc de définir comme groupes témoins les groupes pour lesquels les notions abordées l'ont été au travers de cette appropriation classique.


\subsubsection{Types et valeurs de base : groupes témoins}
Cette phase a été réalisée au sein de deux groupes témoins au lycée Monte-Cristo d'Allauch.

Il s'agissait de la première séquence de l'année, au cours de laquelle les élèves découvraient l'informatique en plus de leur enseignant et leurs camarades. La méthode \og classique \fg{} se retrouve dans la construction générale de la séquence, même si les parties cours et TD étaient quelque peu mélangées.

Cette séquence a duré un mois, période après laquelle la première évaluation sommative a été réalisée.

Il s'agissait tout d'abord de travailler au cours des premières séances sur le dénombrement, afin de pouvoir aborder l'écriture d'un entier en binaire. Le but était de faire prendre conscience aux élèves que le dénombrement décimal utilisé tous les jours n'est qu'une représentation comme une autre de nombres, qui peuvent ainsi être représentés différemment, comme en binaire ou en hexadécimal. C'est comme cela que ces représentations ont été introduites.

Par la suite, un document de cours contenant l'ensemble des notions de la séquence a été distribué, et chaque contenu a été abordé au cours de séances de TD pendant 4 à 6 heures chacun.

\subsubsection{Constructions élémentaires : groupe témoin}
Cette phase a été réalisée au sein d'un groupe témoin au lycée Simone Veil de Marseille.