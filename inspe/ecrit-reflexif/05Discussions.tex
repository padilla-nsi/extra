QCM $\rightarrow$ Apprendre\\
Comprendre ?\\
Avec le QCM, a-t-on bien évalué la connaissance de l'élève ?\\
Les questions sont-elles toutes du même ordre de difficulté pour chaque notion évaluée ? L'analyse qui en est faite rend-elle compte du niveau réel de compréhensions/d'apprentissage des élèves ?


Collaboration $\rightarrow$ Sur le long terme les élèves gagnent en confiance en eux car ils se sentent inclus au sein du groupe (sans jugement des autres ?)\\


Évaluations avant séquence et un mois après $\rightarrow$ Diagnostic des connaissances mais pas de compétences ?


Effet nouveauté $\rightarrow$ Le fait de ne mettre en place Jigsaw que pour une seule séance ne permet pas aux élèves de s'habituer au dispositif, donc peut-être que cela rend nos données moins pertinentes...

Pertinence des choix des notions du programme pour lesquels on propose Jigsaw ? Est-ce que ces notions sont propices au travail de groupe ?

Interdépendance fonctionnelle non suscitée dans la deuxième phase de Jigsaw : la collaboration a été remplacée par la coopération, car l'organisation de la tâche a permis une simple répartition de celle-ci qui ne soit qu'une addition de compétence individuelle.