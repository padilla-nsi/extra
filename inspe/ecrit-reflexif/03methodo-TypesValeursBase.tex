\newpage

\subsection{Types et valeurs de base : groupe test}


Cette phase a été réalisée au sein d'un groupe test au lycée Simone Veil de Marseille.


\subsubsection{Savoir abordés}

Les \textbf{savoirs} abordés lors des différents moments de notre dispositif sont les suivants:

\begin{tabularx}{\linewidth}{|c|p{6cm}|Y|Y|} \hline
    \bfseries Savoir & \centering\bfseries Description & \bfseries Moment de différenciation & \bfseries Moments appropriation et échange 
    \\ \hline
    S1 & écriture d'un entier positif dans une base $b=2$ & $\times$ & 
    \\ \hline
    S2 & écriture d'un entier positif dans une base $b>2$ & & $\times$
    \\ \hline
    S3 & représentation binaire d'un entier relatif & &$\times$
    \\ \hline
    S4 & représentation d'un texte en machine & & $\times$
    \\ \hline
\end{tabularx}



\subsubsection{Moment de la différenciation}


Ce moment s'est déroulé sur une séance et a comme objectifs \emph{pour l'enseignant} de créer trois groupes de niveaux et \emph{pour l'élève} de travailler les prérequis nécessaires permettant d'aborder le travail à venir en groupes autonomes.


\paragraph{Découpage en tâches}
%
Le savoir S1 \emph{écriture d'un entier positif dans
une base b = 2} peut être découpé en tâches dévolues à l'élève. Ces tâches sont réalisées par l'élève au cours de différentes activités.

\begin{tabularx}{\linewidth}{|cXc|} \hline
    \bfseries Tâches & \centering \bfseries Description & \bfseries Activité \\ \hline
    & \centering \bfseries écriture d'un entier naturel en base 10 & \\
    T1.1 & passer d'une puissance de 10 à son écriture en base 10  & Act.1 \\
    T1.2 & passer d'un entier naturel écrit en base 10 à sa décomposition en somme de produit de puissances de 10  & Act.1 \\
    T1.3 & passer d'un entier naturel écrit comme une décomposition en somme de produit de puissances de 10 à son écriture en base 10 & Act.1 \\ 
    \hline
    \hline

    & \centering \bfseries écriture d'un entier naturel en base 2 & \\
    T2.1 & passer d'une puissance de 2 à son écriture en base 10 & Act.2
    \\
    T2.2 & passer d'un entier naturel écrit en base 10 à sa décomposition en somme de produit de puissances de 2 & Act.2\\ 
    \hdashline

    T2.3 & convertir en binaire un entier naturel écrit en base 10 & Act.3\\
    T2.4& convertir en base 10 un entier naturel écrit en binaire & Act.3\\
    \hdashline

    T2.5 & écrire en binaire les entiers naturels précédent et suivant un nombre donné écrit en binaire & Act.4 \\
    T2.6 & écrire en binaire le double et la moitié d'un nombre donné écrit en binaire & Act.4 \\
    T2.7 & écrire en binaire la somme de deux entiers naturels donnés écrits en binaire & Act.4\\
    \hline
\end{tabularx}


\paragraph{Mise en place des activités}
%
La séance a été découpée en quatre activités créées et déployées sur la plateforme de formation en ligne Moodle\footnote{\href{https://moodle.org}{https://moodle.org}} fournie par le rectorat.
\\
Le choix d'une telle plateforme nous permet \textbf{d'engager fortement les élèves} par le biais de valeurs aléatoires et de rétroaction \textbf{de mettre en place une mesure} de l'autonomie et du niveau d'acquisition des savoirs par les élèves.

\begin{description}
    \item[Act.1] écriture décimale et les puissances de 10
    \item[Act.2] puissances de 2
    \item[Act.3] conversions entre les bases décimales et binaires et 
    \item[Act.4] calculs avec des nombres entiers représentés en binaires.
\end{description}

\paragraph{Évaluer les élèves}%
%
Évaluer la réussite des élèves est simple et rapide avec Moodle. Il suffit de récupérer pour chaque tâche, la note obtenue par l'élève. Afin d'engager l'élève un maximum tout en l'incitant à réussir chacune des tâches proposées, nous avons autorisé un nombre quelconque de tentatives. Pour chacune d'entre elle, une rétroaction indique son niveau de réussite et propose une correction en lien avec les valeurs aléatoires de l'énoncé.

Évaluer le degré d'autonomie de l'élève nous a demandé une analyse plus fine des données récoltées. Nous avons décidé de mesurer la durée totale nécessaire à un élève pour réaliser une tâche donnée. Cette mesure de rapidité de l'élève nous semble un bon indicateur de son degré d'autonomie.
\\
Pour créer les groupes de niveaux, nous avons, sur chaque activité, classé les élèves du plus rapide au moins rapide. Nous avons alors attribué à tous les élève un nombre de point égal à son classement. Une activité achevée sur son temps personnel se voit attribuée 15 points et une activité inachevée 30 points.
Enfin, nous avons additionné tous ces points et obtenu un total dont les valeurs les plus petites sont associées aux élèves les plus rapides ayant fait le plus d'exercices. De façon symétrique, les valeurs les plus grandes sont associées aux élèves les moins rapides ayant terminé des exercices sur temps personnel.


Voici le tableau anonymisé des scores obtenus à partir de la séance réalisée avec 13 élèves présents. L'élève s'est vu attribué le score maximal. 

\begin{center}
    \begin{tabular}{|c|c|c|c|c|c|}\hline
    \bfseries Élève & \bfseries Activité 1
    & \bfseries Activité 2 & \bfseries Activité 3
    & \bfseries  Activité 4 & \bfseries  TOTAL \\ \hline
    \rowcolor{yellow!30} élève 1 & 7 & 3 & 1 & 1 & 12 \\ \hline
    \rowcolor{yellow!30} élève 2 & 1 & 5 & 5 & 15 & 26 \\ \hline
    \rowcolor{yellow!30} élève 3 & 6 & 7 & 2 & 15 & 30 \\ \hline
    \rowcolor{yellow!30} élève 4 & 3 & 1 & 4 & 30 & 38 \\ \hline
    \rowcolor{orange!30} élève 5 & 4 & 6 & 15 & 15 & 40 \\ \hline
    \rowcolor{orange!30} élève 6 & 9 & 2 & 3 & 30 & 44 \\ \hline
    \rowcolor{orange!30} élève 7 & 8 & 9 & 15 & 15 & 47 \\ \hline
    \rowcolor{orange!30} élève 8 & 5 & 8 & 6 & 30 & 49 \\ \hline
    \rowcolor{orange!30} élève 9 & 10 & 12 & 15 & 15 & 52 \\ \hline
    \rowcolor{orange!60} élève 10 & 11 & 11 & 15 & 15 & 52 \\ \hline
    \rowcolor{orange!60} élève 11 & 13 & 13 & 15 & 15 & 56 \\ \hline
    \rowcolor{orange!60} élève 12 & 2 & 4 & 30 & 30 & 66 \\ \hline
    \rowcolor{orange!60} élève 13 & 12 & 10 & 30 & 30 & 82 \\ \hline
    \rowcolor{red!60} élève 14 & 30 & 30 & 30 & 30 & 120 \\ \hline
    \end{tabular}    
\end{center}

Cette évaluation des élèves nous permet de créer trois groupes de niveaux. Les élèves 1 à 4 auront les savoirs les plus complexes à étudier en autonomie, les élèves 5 à 9 les savoirs de niveau intermédiaire et les élèves 10 à 14 les savoirs les plus simples.

% , nous avons décidé que les activités proposées se réaliseraient de façon autonome. Nous avons alors mesuré la durée qu'il a fallu à l'élève pour réaliser une tâche donnée.

% Outre le niveau de réussite des tâches, nous avons noté la rapidité avec laquelle elles ont été réalisées. D'après nous cette mesure est significative du degré d'autonomie de l'élève ainsi que de son aisance avec le savoir-faire travaillé.

% Pour chaque activité, nous avons donc classé les élèves par ordre d'achèvement. L'élève terminant l'activité le premier se voit attribué le coefficient 1, le deuxième élève le coefficient 2, etc. Les élèves ayant terminés l'activité en dehors de la classe se voient attribués le coefficient 15 et ceux n'ayant pas terminé l'activité le coefficient 30.

% En additionnant tous les coefficients, nous obtenons pour chaque élève un total. Nous avons enfin regroupés ensemble les 4 totaux les plus élevés (pour le groupe 1, niveau facile), les 4 totaux les plus faibles (pour le groupe 3, niveau difficile) et les 5 autres totaux intermédiaires (pour le groupe 2, niveau intermédiaire).

% D'abord puisque cet outil est une \emph{plateforme de formation}, nous pouvons obtenir de nombreux indicateurs relatifs au travail des élèves. 

% Ensuite, puisque cet outil est \emph{en ligne}, les élèves peuvent procéder à de nombreuses tentatives. Les exercices proposés sont à chaque fois différents sur les valeurs numériques, mais identiques sur la méthode de résolution. Les élèves peuvent donc être dans une démarche d'appropriation basée sur un cycle d'essais et de rétroactions avec un corrigé personnalisé. Nous avons ainsi pu observer un engagement fort de la part de chaque élève (par exemple certaines activités ont été recommencées 4 fois).

% \paragraph{Déroulement des activités}
% %
% Les tâches essentielles ont été réalisées par 85\% des élèves.
% L'analyse du travail des élèves nous a permis d'évaluer le niveau de réussite en autonomie des élèves.

% Les tâches 1.1, 1.2, 1.3, 2.1 et 2.2 ont été réalisées entière en classe par tous les élèves.

% Les tâches 2.3 et 2.4 ont été :
% \begin{itemize}
%     \item réalisées entièrement (en classe) par 6 élèves sur 13,
%     \item réalisées entièrement (en classe puis hors classe) par 5 élèves sur 13 et
%     \item réalisées partiellement par 2 élèves sur 13.
% \end{itemize}

% Les tâches 2.5, 2.6 et 2.7 ont été :
% \begin{itemize}
%     \item réalisées entièrement (en classe) par 1 élève sur 13,
%     \item réalisées entièrement (en classe puis hors classe) par 7 élèves sur 13 et 
%     \item non abordées par 5 élèves sur 13.
% \end{itemize}


% Cette première analyse permet de mettre en évidence les deux élèves les plus en difficultés.




\begin{description}
    \item[groupe Appropriation 1, niveau difficile] Savoir S4 : la représentation d'un texte en machine
    \item[groupe Appropriation 2, niveau intermédiaire] Savoir S3 : la représentation binaire des entiers relatifs.
    \item[groupe Appropriation 1, niveau facile] Savoir S2 : l'écriture d'un entier naturel en base 16.
\end{description}

\newpage