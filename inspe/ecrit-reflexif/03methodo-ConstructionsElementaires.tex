
\subsection{Constructions élémentaires : groupes tests}
Cette phase a été réalisée au sein de deux groupes témoins au lycée Monte-Cristo d'Allauch.

\idee{Les contenus de programme étudiés lors des trois moments didactiques de notre méthodologie.}
Conformément à notre méthodologie, la séquence est partagée en trois moments didactiques distincts : le moment de la différenciation, le moment de l'appropriation et le moment de l'échange. 

Chacun de ces moments est l'occasion pour l'élève d'aborder les contenus de programme identifiés :

\begin{description}
    \item[Moment de la différenciation] qui permet à l'enseignant de créer trois groupes de niveau et à l'élève de (re)travailler les prérequis nécessaires lors des moments suivants :
    \begin{itemize}
        \item Contenu 1
    \end{itemize}
    \item[Moment de l'appropriation] qui permet à l'élève d'aborder un contenu adapté à son niveau parmi les trois possibles :
    \begin{itemize}
        \item Contenu 2 (pour un élève du groupe 1, niveau facile)
        \item Contenu 3 (pour un élève du groupe 2, niveau intermédiare)
        \item Contenu 4 (pour un élève du groupe 3, niveau difficile)
    \end{itemize}
    \item[Moment de l'échange] qui permet à l'élève d'aborder les deux contenus du moment précédent qu'il n'a pas encore rencontré.
\end{description}

Les activités proposées dans ce groupe pour le moment de différenciation sont les suivantes : 

\begin{description}
    \item[activité 1] traductions d'expression écrites en français en expressions booléennes, et inversement
    \item[activité 2] analyses de séquences d'affectations de variables
    \item[activité 3] écritures d'algorithmes simples mettant en œuvre des affectations et 
    \item[activité 4] analyses d'algorithmes et écritures de types de données.
\end{description}

Les tâches réalisées pendant le moment de la différenciation sont les suivantes :

\begin{description}
    \item[Activité 1] traductions d'expression écrites en français en expressions booléennes, et inversement
    \begin{description}
        \item[Tâche 1.1] écritures d'expressions booléennes à partir d'expressions mathématiques en français
        \item[Tâche 1.2] écritures en français d'expressions booléennes algorithmiques
        \item[Tâche 1.3] écritures d'expressions booléennes à partir d'expressions non mathématiques en français
    \end{description}

    \item[Activité 2] analyses de séquences d'affectations de variables
    \begin{description}
        \item[Tâche 2.1] écriture de valeurs finales de variables à partir de lecture d'algorithmes 
        \item[Tâche 2.2] analyse de valeurs finales de variables à partir de la lecture d'un mauvais algorithme
        \item[Tâche 2.3] écriture de l'algorithme répondant au problème de la tâche précédente
        \item[Tâche 2.4] écriture d'une extension/généralisation de l'algorithme de la tâche précédente
    \end{description} 

    \item[Activité 3] écritures d'algorithmes simples mettant en œuvre des affectations
    \begin{description}
        \item[Tâche 3.1] écriture et lecture de valeurs de variables 
        \item[Tâche 3.2] écriture d'un algorithme mathématique simple (calcul du cube d'un nombre)
        \item[Tâche 3.3] écriture d'un algorithme de la vie courante (calcul de TVA)
    \end{description}

    \item[Activité 4] analyses d'algorithmes et écritures de types de données
    \begin{description}
        \item[Tâche 4.1] lecture et analyse d'algorithmes mettant en œuvre des affectations et des affichages
        \item[Tâche 4.2] analyse de types de données à partir de la lecture d'un algorithme plus complexe
    \end{description} 
\end{description}


\idee{Partage des différentes tâches entre les activités}
Cette décomposition en tâche permet donc de clarifier l'objectif de chacune des activités du moment de la différenciation.

\info{\#TODO faire un tableau ?}
\begin{description}
    \item[Activité 1] :
    \begin{itemize}
        \item Tâche 1.1
        \item Tâche 1.2
        \item Tâche 1.3
    \end{itemize}
    
    \item[Activité 2] :
    \begin{itemize}
        \item Tâche 2.1
        \item Tâche 2.2
    \end{itemize}
    
    \item[Activité 3] :
    \begin{itemize}
        \item Tâche 2.3
        \item Tâche 2.4
    \end{itemize}

    \item[Activité 4] :
    \begin{itemize}
        \item Tâche 2.5
        \item Tâche 2.6
        \item Tâche 2.7
    \end{itemize}
\end{description}

%
%\idee{Deux intérêts d'utiliser une plateforme de formation en ligne.}
%Ces activités ont été créées sur la plateforme de formation en ligne Moodle\footnote{\href{https://moodle.org}{https://moodle.org}}. Ce choix technique a de nombreux avantages. 
%
%D'abord puisque cet outil est une \emph{plateforme de formation}, nous pouvons obtenir de nombreux indicateurs relatifs au travail des élèves. 
%
%Ensuite, puisque cet outil est \emph{en ligne}, les élèves peuvent procéder à de nombreuses tentatives. Les exercices proposés sont à chaque fois différents sur les valeurs numériques, mais identiques sur la méthode de résolution. Les élèves peuvent donc être dans une démarche d'appropriation basée sur un cycle d'essais et de rétroactions avec un corrigé personnalisé. Nous avons ainsi pu observer un engagement fort de la part de chaque élève (par exemple certaines activités ont été recommencées 4 fois).
%
%
%\idee{Les tâches essentielles ont été réalisées par 85\% des élèves.}
%L'analyse du travail des élèves nous a permis d'évaluer le niveau de réussite en autonomie des élèves.
%
%Les tâches 1.1, 1.2, 1.3, 2.1 et 2.2 ont été réalisées entière en classe par tous les élèves.
%
%Les tâches 2.3 et 2.4 ont été :
%\begin{itemize}
%    \item réalisées entièrement (en classe) par 6 élèves sur 13,
%    \item réalisées entièrement (en classe puis hors classe) par 5 élèves sur 13 et
%    \item réalisées partiellement par 2 élèves sur 13.
%\end{itemize}
%
%Les tâches 2.5, 2.6 et 2.7 ont été :
%\begin{itemize}
%    \item réalisées entièrement (en classe) par 1 élève sur 13,
%    \item réalisées entièrement (en classe puis hors classe) par 7 élèves sur 13 et 
%    \item non abordées par 5 élèves sur 13.
%\end{itemize}
%
%Cette première analyse permet de mettre en évidence les deux élèves les plus en difficultés.
%
%
%\idee{Une mesure simple qui permet de coupler le niveau d'acquisition et le degré d'autonomie de l'élève.}
%Outre le niveau de réussite des tâches, nous avons noté la rapidité avec laquelle elles ont été réalisées. D'après nous cette mesure est significative du degré d'autonomie de l'élève ainsi que de son aisance avec le savoir-faire travaillé.
%
%Pour chaque activité, nous avons donc classé les élèves par ordre d'achèvement. L'élève terminant l'activité le premier se voit attribué le coefficient 1, le deuxième élève le coefficient 2, etc. Les élèves ayant terminés l'activité en dehors de la classe se voient attribués le coefficient 15 et ceux n'ayant pas terminé l'activité le coefficient 30.
%
%En additionnant tous les coefficients, nous obtenons pour chaque élève un total. Nous avons enfin regroupés ensemble les 4 totaux les plus élevés (pour le groupe 1, niveau facile), les 4 totaux les plus faibles (pour le groupe 3, niveau difficile) et les 5 autres totaux intermédiaires (pour le groupe 2, niveau intermédiaire).
%
%
%\begin{description}
%    \item[groupe Appropriation 1, niveau facile] Contenu 3 : l'écriture d'un entier naturel en base 16,
%    \item[groupe Appropriation 2, niveau intermédiaire] Contenu 4 : la représentation binaire des entiers relatifs
%    \item[groupe Appropriation 3, niveau difficile] Contenu 5 : la représentation d'un texte en machine
%\end{description}
