Ordonnancement des idées :
\begin{enumerate}
	\item Groupes de spécialité et NSI
	\item Premier questionnement
	\item Groupe classe mal défini
	\item Pourtant socio-constructivisme
	\item Pédagogie différenciée
	\item Catégorisation de la pédagogie différenciée
	\item Notre dispositif
\end{enumerate}



La formulation explicite de notre question de recherche est le fruit d'une maturation lente. Dans cette partie, nous détaillons le cheminement de notre réflexion.


\subsection{Cadre socio-constructiviste}


%\idee{L'enseignement contemporain se fait dans un cadre constructiviste.}
\paragraph{Constructivisme}
Tout d'abord, notre mémoire se place dans le cadre du \emph{constructivisme}.
%
% \info{reformulation : ok}
%
Développée dès les années 1920 par Jean Piaget, qui la synthétise dans un de ses derniers ouvrages \citep{piaget_prise_1974}, cette théorie de l'apprentissage part du principe qu'un savoir ne se transmet pas ipso facto par l'enseignant, mais qu'il se construit par l'apprenant. Là où des théories comme le béhaviorisme prônaient l'inné face à l'acquis et l'apprentissage comme un réflexe face à un stimuli, le constructivisme propose quant à lui d'étudier l'activité du sujet en train de construire sa représentation mentale de la réalité. 
\\
Pour nous, cela se traduit par l'évidence qu'un cours n'est rien s'il ne permet pas une mise en activité de l'élève. Expérimenter, chercher, questionner, formuler, comprendre plutôt qu'apprendre, écouter ou noter.
%
D'ailleurs nos référentiels n'expriment pas autre chose et dire de l'élève qu'il est un acteur actif au centre de son apprentissage est désormais un point de vue que nous imaginons bien ancré et assimilé par le corps enseignant.



%\idee{Notre pratique quotidienne est ancrée dans un cadre socio-constructiviste.}
\paragraph{Socio-constructivisme}
Pour être plus précis, notre mémoire se place dans un cadre \emph{socio-constructiviste}.
%
\info{reformulation : OK\\Ajouter Vygotsky ! $\rightarrow$  Garnier, C., Bednarz, N. \& Ulanovskaya, I. (2009). Après Vygotski et Piaget: Perspectives sociale et constructiviste. Louvain-la-Neuve: De Boeck Supérieur.}
%
% Confronter les points de vue pour améliorer le sien et améliorer ses connaissances.
% Un tiers en réussite, un tiers moyen, un tiers en difficulté -> hétérogénéité "parfaite"
% Classes de niveau -> échec
% Vygotsky ? -> ZPD
% Roux, 1999
% Piaget : "Réussir et comprendre, c'est 2 choses différentes"
Découlant des travaux de Piaget, ce cadre a été développé par Willem Doise et Gabriel Mugny en \citeyear{doise_developpement_1981} et  \citeyear{doise_psychologie_1997}. Cette théorie se propose d'élargir l'environnement de l'apprenant en y intégrant la dimension sociale. Les interactions entre élèves y sont vues comme un instrument pour l'enseignant qu'il peut mettre à profit pour une mise en action favorisant l'apprentissage, car confronter les points de vue permettrait d'améliorer le sien et donc d'améliorer ses connaissances.
\\
Malgré tout notre professionnalisme et notre enthousiasme, il faut bien reconnaître que si un élève contemporain tolère l'obligation d'aller à l'école, c'est d'abord et surtout pour y retrouver des camarades, échanger et partager plutôt que pour y apprendre, travailler et progresser\footnote{Affirmation issue de notre expérience personnelle d'enseignant \textbf{et} d'élève.}.
%
Notre enseignement ne peut négliger ce besoin naturel de sociabilisation et notre travail de recherche s'articule donc tout naturellement autour de pratiques qui mettent en avant et valorisent les interactions entre élèves.
\\
Le socio-constructivisme se base également sur les travaux de Lev Vygotski, pédagogue russe du début du XX\up{ème} siècle ayant théorisé le concept de \textit{Zone Proximale de Développement (ZPD)}, qui définit l'espace entre lequel un enfant sait résoudre une tâche tout seul (en autonomie) et sait la résoudre avec une aide. La principale difficulté soulevée par cette \textit{ZPD} réside dans le fait qu'une personne avancée dans un domaine doit savoir entrer dans la \textit{ZPD} de l'apprenant pour que ce dernier puisse apprendre. Ce n'est qu'en recouvrant les \textit{ZPD} de l'apprenant et de l'enseignant que l'apprenant peut apprendre.




\subsection{Cadre de la différenciation}

En plus du socio-constructivisme, notre mémoire se place dans le cadre des pédagogies différenciées.

\subsubsection{Nécessité d'une pédagogie différenciée}\label{Pédagogie différenciée}

%\idee{Un constat : pour l'institution, la pratique d'une pédagogie différenciée est nécessaire.}
\paragraph{Nécessité issue du monde institutionnel}
%
% \info{reformulation : ok}
%
En premier lieu, les références institutionnelles sur lesquels nous devons nous appuyer en tant enseignants en France pour construire nos enseignements demandent explicitement de différencier. Ces textes imposent aux enseignants des valeurs communes, comme par exemple la laïcité, et des pratiques communes comme la mise en place pour les élèves d'\textcite{un accompagnement pédagogique adapté aux besoins de chacun afin de favoriser la réussite de leur scolarité [\ldots] Des propositions de différenciations doivent permettre à chaque élève de maîtriser les compétences attendues} \citep{eduscol_accompagnement_2020}.
\\
Cette demande explicite de différenciation dans notre pédagogie concerne aussi bien le champ du handicap que celui de la psychologie ou encore du social et s'appuie sur un socle théorique fort.



%\idee{Observant des différences de rythmes entre enfants, certains pédagogues proposent de privilégier l'accompagnement individuel.}
\paragraph{Nécessité issue du monde de la recherche}
%
% \info{reformulation : ok}
%
Évoquons maintenant le monde de la recherche. Dès lors qu'ils se sont intéressés à l'apprenant, les chercheurs en pédagogie n'ont pu que constater les différences entres individus.
Ainsi dans la première moitié du XX\up{ème} siècle, Piaget parvient à définir quatre stades de développement de l'enfant. Il observe que ces stades dépendent de tranches d'âge mais n'interviennent pas à la même période pour chaque enfant. 
Déjà la question des différences de rythmes de développement et de compréhensions entre élèves est posée.
%
De son côté, Celestin Freinet prône des pédagogies centrées sur l'enfant plutôt que sur l'enseignant ce qui amène ses contemporains à comprendre qu'\textcite{il n'est pas possible de proposer des apprentissages uniformes pour tous les élèves d'une même classe}, et qu'\textcite{il faut plutôt se diriger vers des apprentissages semblables.} \citep{lautru_pedagogie_2019}



%\idee{La formalisation du concept de pédagogie différenciée comme un effort pour répondre à la diversité des élèves.}
\paragraph{Formalisation du concept}
%
% \info{reformulation : ok}
%
La formalisation du concept de pédagogie différencié n'est cependant apparue que dans les années 1970. Dans ses travaux, Louis Legrand la définit comme \textcite{un effort de diversification méthodologique susceptible de répondre à la diversité des élèves} \citep{legrand_differenciation_1986}.
\\
En pratique, la pédagogie différenciée consisterait selon lui à \textcite{utiliser toutes les ressources possibles pour permettre aux élèves de développer leurs connaissances, ce qui suppose la diversité des démarches mais aussi des outils} \citep{battut_comment_2009}.
%
Ainsi, l'enseignant doit pouvoir proposer à chaque élève des tâches différentes afin que chacun puisse acquérir les notions issues des programmes officiels. 



% \idee{Le concept de pédagogie différenciée concerne aussi les diversités entre collectifs comme les classes.}
% Pointer le fait qu'il y a la théorie et la pratique, que tout ça est difficile à mettre en œuvre au quotidien car les différences existent tout le temps et pour tout le monde, que ce soit au niveau individuel (différences de niveau, de compréhension) comme au niveau collectif (les classes sont toutes différentes entre elles) 



%\idee{La pédagogie différenciée est naturellement présente dans toute situation d'enseignement.}
\paragraph{Situation actuelle}
%
% \info{reformulation : ok}
%
De nos jours, la \textcite{pédagogie différenciée} apparaît comme évidente et il nous semble naturel de la voir présente dans toute situation d'enseignement. \textcite{[Elle] est une réalité quotidienne incontestable} \citep{perrenoud_pedagogie_1995} et \textcite{est toujours là d'abord, avant tout effort pédagogique particulier, comme une réalité sociologique observable} \citep{meirieu_pedagogie_1996}.



\subsubsection{Mise en action de la pédagogie différenciée}



%\idee{Parce qu'elle est par essence centrée sur la différence entre individus, la méthode unique et universellement efficace de pédagogie différenciée n'existe pas.}
\paragraph{Pas de méthode universelle}
%
% \info{reformulation : ok}
%
Dans ses travaux sur la pédagogie différenciée, \cite{perrenoud_pedagogie_1997} affirme que \textcite{le cercle de ceux qui y réfléchissent et tentent quelque chose s'élargit}. On ne peut que constater qu'\textcite{une large palette de démarches et de procédés [\ldots] pour que les élèves apprennent un ensemble de savoirs et de savoir-faire commun à tous} \citep{battut_comment_2009} est proposée.
\\
Malgré ce foisonnement, Philippe \cite{meirieu_pedagogie_1996} se demande \textcite{pourquoi [il est] si difficile de mettre en pratique ses convictions pédagogiques}. Il identifie et oppose alors \textcite{deux grands courants théoriques de la différenciation} :
%
\begin{itemize}
	\item le \textcite{diagnostic a priori} : courant dans lequel \textcite{l'éducateur cherche à atteindre une sorte de \emph{nature profonde} du sujet qui lui permet de le classer dans une catégorie pour laquelle il dispose d'un ensemble de solutions} et 
	\item l'\textcite{inventivité régulée} : courant dans lequel l'\textcite{[éducateur] prend des indices qui lui permettent seulement de statuer sur les besoins du moment et d'avancer une proposition particulière dont on ne sait jamais d'avance comment elle sera accueillie et quels effets elle produira}.
\end{itemize}
%
Dans cette catégorisation, nous retrouvons encore la distinction entre le béhaviorisme et le constructivisme initiée par Piaget. Tout comme son prédécesseur, Meirieu conclut que pour lui l'\textcite{inventivité régulée} est la méthode à suivre car plus \textcite{ouverte}.
%
Malgré cette indication, il ne pousse pas plus vers l'opérationnalisation de la pédagogie différenciée et ne propose aucune méthode permettant sa mise en pratique. Il analyse cependant que \textcite{le pédagogue \emph{[est]} enrichi ou meurtri de ses expériences passées}, et que c'est donc par l'expérience pratique et théorique que l'enseignant pourra trouver la solution au problème pédagogique de chaque moment.



%\idee{L'enseignant doit utiliser en classes des dispositifs diversifiés favorisant les interactions entres les élèves.}
\paragraph{Concrétisation et limites}
%
% \info{reformulation : ok}
%
\cite{perrenoud_pedagogie_1997} fait partie des chercheurs contemporains qui s'intéressent à la mise en œuvre concrète d'une pédagogie différenciée dans les salles de classe. Ses publications portent donc sur la façon d'outiller l'enseignant afin de lui permettre d'acquérir une expérience pratique étayée par des apports théoriques. Selon lui, \textcite{le vrai défi est d'imaginer les dispositifs favorisant des interactions entre élèves, dans le cadre de divers groupes de travail, sans empêcher une individualisation du parcours de chacun}.\info{battut ; perrenoud}

%
%\idee{Pratiquer la pédagogie différenciée sans hésiter à  abandonner ses objectifs les plus ambitieux.}
%
De nombreux freins existent quant à la pratique en classe de la différenciation pédagogique. Par exemple en privilégiant l'accompagnement individuel,  le risque existe pour l'enseignant de baisser ses exigences collectives afin de permettre au plus grand nombre d'avancer dans l'apprentissage. Mais avec sa volonté forte de rendre opérationnelle la pédagogie différenciée, Philippe Perrenoud propose de se déculpabiliser. Son conseil est ainsi de \textcite{tourner le dos aux objectifs les plus ambitieux, pour assurer au moins l'égalité des acquis minimaux}.




%\idee{L'individualisation peut avoir des effets pervers.}
%
% \info{reformulation : ok}
%
La mise en pratique de la pédagogie différenciée doit aussi s'accompagner de précautions. Dans ses écrits, \cite{feyfant_individualisation_2008} promeut une pédagogie différenciée qualifiée d'\emph{individualisée} tout en mettant en garde les enseignants sur \textcite{l'individualisation [qui] peut prendre de multiples formes et avoir des effets bénéfiques ou à l'inverse stigmatiser et creuser les différences et les inégalités}. Dans un écrit plus récent, Annie Feyfant propose encore une autre précaution qui est de conduire une différenciation pour l'élève et non pour l'enseignant. Ainsi, l'\textcite{l'objectif n'est pas tant de différencier \emph{en soi} que la nécessité d'accompagner au mieux les élèves dans leurs apprentissages} \citep{feyfant_differenciation_2016}.




\subsection{Groupes de spécialité : individualisation et individualisme}


%\idee{Un constat : les groupes d'enseignement de spécialité sont constitués d'élèves provenant de classes différentes.}
\info{À revoir si on le déplace}Après avoir spécifié notre cadre d'étude, intéressons-nous maintenant à notre discipline : la \gls{nsi}.
\paragraph{Enseignement de spécialité et morcellement des classes}
%
% \info{reformulation : ok}
%
Depuis 2018 et la \emph{réforme du baccalauréat général et technologique et du lycée}, les enseignements de spécialités ont vu le jour, dont celui de \gls{nsi}. Pour son passage en classe de première, en dehors des disciplines du tronc commun, un élève de lycée général doit désormais choisir trois spécialités parmi l'ensemble des enseignements disponibles sur son secteur d'affectation. Cette individualisation du choix couplée à la richesse de l'offre engendre une très grande variété de couplages sur les cohortes d'élèves de première et de terminale\footnote{Par exemple avec 9 spécialités proposées sur un lycée, il y a théoriquement 84 combinaisons possibles de spécialité en première.}. Sur ces deux niveaux, hormis quelques exceptions et par volonté institutionnelle de ne pas recréer des classes de spécialité (quand bien même plusieurs élèves suivraient les mêmes spécialités), il n'existe donc plus de classe aux spécialités homogènes. C'est-à-dire qu'il n'existe plus de classes dont tous les élèves suivent les mêmes enseignements de spécialité. Les classes de spécialité sont donc devenues des \emph{groupes} de spécialités, et la très grande majorité de ces groupes sont constitués d'élèves provenant de classes différentes.



%\idee{L'individualisation induite par un groupe de spécialité constitué d'une somme d'individus qui ne partagent pas d'autres moments ensembles nous pose problème.}
\paragraph{Individualisation des parcours}
%
% \info{reformulation : ok}
%
Ce morcellement des classes implique que les élèves se retrouvant dans chaque groupe de spécialité se connaissent beaucoup moins. Ils ne se côtoient ensemble que pendant les heures consacrées à chaque enseignement de spécialité.  Le parcours de chaque élève est donc fortement individualisé et les groupes de spécialité sont structurellement d'une constitution autre que celle de la classe entière ou de ses demi-groupes.
\\
Cette individualisation des parcours et ce cloisonnement de l'élève nous interpellent fortement.
%
D'abord le travail extra-scolaire des élèves est fortement impacté. Prenons l'exemple de la réalisation d'un projet par des élèves. L'implication du groupe de projet en dehors des heures d'enseignement de spécialité est fortement entravée pour de simples raisons d'organisation, car les emplois du temps diffèrent et les créneaux libres ne se chevauchent plus.
%
Ensuite, cette diminution du temps personnel que nous venons d'évoquer a pour conséquence qu'un élève studieux aura moins de temps pour ce qui aurait été auparavant dévolu à des activités extra-scolaires lui permettant de se construire humainement et/ou de décompresser de la charge mentale du temps scolaire.
%
Enfin, dans notre société qui exacerbe l'individualisation, il nous semble essentiel de travailler sur la promotion du collectif. À ce titre, il nous apparaît que le fonctionnement structurel induit par la \emph{réforme du baccalauréat général et technologique et du lycée} engendre une école cloisonnant l'individu plutôt qu'une école comme levier de promotion du collectif. 

Comme le dit Gilles Monceau au sujet de la manière de voir les parcours scolaires et les apprentissages : \textit{aujourd'hui, c'est l'individualisation qui s'impose progressivement comme une nécessité et une évidence} \citep{monceau_groupe_2005}. Mais ce constat est extrêmement déstabilisant pour nous puisque les contenus des référentiels le contredisent et mettent en avant, par exemple, les compétences transversales de collaboration, de travail en équipes ou encore d'échanges !



%\idee{Notre intuition de professeur stagiaire est de vouloir recréer un vécu partagé entre les élèves, comme cela était le cas avant la création des enseignements de spécialité.}
\paragraph{Première intuition}
%
% \info{reformulation : ok}
%
En tant qu'enseignants de spécialité, nous nous retrouvons donc face à un groupe d'élèves dont les seuls moments partagés se font dans notre cours. 
%
Cet état des lieux nous interpelle car il apparaît en contradiction avec le cadre socio-constructiviste. Une première intuition a donc été de nous dire que cette méconnaissance des élèves les uns envers les autres n'est pas bénéfique pour leurs apprentissages. Si les interactions entre les élèves sont minimales, il n'y a alors ni échange, ni partage ou encore absence d'intérêt les uns pour les autres.
%
Au-delà de la mise en activité par le travail scolaire, il nous semble donc essentiel de favoriser aussi les interactions entre les élèves. En ce sens et puisque la notion de classe n'existe plus, notre idée a été de vouloir chercher les moyens de favoriser l'émergence d'un groupe, d'un collectif. Nous imaginons que l'expérience d'un vécu partagé et la connivence entre élèves seront quant à eux favorables aux apprentissages.



\subsection{Premier questionnement}


%Travailler dans un cadre différencié est aussi un thème fort des théories de l'enseignement. Comme nous venons de lele montre notre analyse dans la partie \ref{Pédagogie différenciée}, le monde de la recherche se penche sur cette question depuis près d'un siècle et une conclusion commune émerge qui tient sur l'évidence de construire des enseignements différenciés, quelle que soit la forme qu'ils prennent.




%\idee{Notre questionnement initial est donc de savoir comment concilier les deux constats précédents avec le socio-constructivisme.}
%Mais avant d'étudier plus attentivement cette nécessité d'une pédagogie différenciée, faisons le point sur l'état de notre réflexion initiale. 
%
% \info{reformulation : ok}
%
Après quelques semaines de travail, notre intuition de professeurs stagiaires nous est apparue antagoniste avec les deux constats précédents. C'est-à-dire que vouloir promouvoir le commun et le collectif pour un meilleur apprentissage nous a semblé contradictoire avec les évidences que (1) les groupes de spécialité sont formés d'élèves provenant de classes différentes et que (2) les institutions, tout comme la recherche, promeuvent la pédagogie différenciée.
%
C'est ainsi que notre questionnement s'est dans un premier temps formulé ainsi  : \emph{comment proposer une pédagogie différenciée à des élèves qui ne se connaissent pas, tout en partageant et échangeant collectivement ?}





\subsection{Le groupe classe, c'est quoi ?}
% Travail de groupe : Environnement socio-cognitif (cerveaux en interaction) susceptible de générer des progrès individuels.
%-> Effets positifs sur la dynamique de chaque individu
%-> Effets positifs sur la représentation de la tâche (on se perd moins).
%-> Déstabilise les procédures résolutoires individuelles (permet de ne pas s'enfermer dans sa procédure). Attention à ne pas trop guider pour permettre cette déstabilisation.
%-> Effets positifs sur le contrôle de l'activité.
%-> Confrontations efficaces
% Vygotsky
% Roux, 1999
% Baudrit, 2007

% Formes de travaux de groupe : collectif, collaboratif, coopératif, tutorat entre pair
% -> Attention à la constitution, au rôle et à la tâche du groupe.

% Constitution et rôle : interdépendance fonctionnelle (collaboration plutôt que coopération car besoin de mise en commun plutôt que somme d'individualité), hétérogénéité mesurée (des groupes de niveaux différents mais pas de trop gros écarts -> les zones proximales de développement doivent se superposer, cad qu'il faut que chacun puisse savoir se mettre au niveau de l'autre), égalité des statuts (un rôle pour chaque membre pour que chacun trouve sa place, sans "chef").
% Exemple de contrainte : que des classes différentes au sein du groupe.

% Temps consacré au travail de groupe ?
% Comment susciter la mise en commun ?

%\idee{La notion de groupe classe est mal définie par la recherche.}
\paragraph{Recherche d'une définition}
%
% \info{reformulation : ok}
Pour répondre à ce questionnement initial, et plus particulièrement pour formaliser le support de l'échange collectif entre élèves, nous avons mené nos premières recherches autour de la notion de \emph{groupe classe}. Contrairement à nos attentes, il se trouve que cette notion est assez mal définie et peu étudiée par le monde de la recherche. Malgré un usage très courant, nos recherches documentaires académiques et institutionnelles se sont vite montrées infructueuses.
\\
Lorsque \cite{peeters_contribution_2018} s'interroge sur la gestion des situations difficiles pour l'enseignant, il détaille les différents niveaux de fonctionnement d'un groupe ainsi que les moyens d'améliorer l'esprit de groupe. Mais il ne définit pas ce qu'est un groupe classe.
%
Même si les questionnements de \cite{monceau_groupe_2005} autour de la sociabilisation engendrée par des regroupements d'élèves en classes et ses réflexions politiques issues de ces questionnements nous ont semblé intéressants, il n'est pas question dans son article \emph{Groupe classe et groupes dans la classe} de définir formellement ce qu'est un groupe classe.
%
%: \textcite{L'hétérogénéité des classes [\ldots] assure de meilleures conditions de réussite pour les élèves les plus en difficulté scolaire [et pourtant] l'École a également produit, au fil de son histoire, des classes dites \emph{spéciales} qui reçoivent les élèves en difficulté dans les classes ordinaires. [\ldots] Bien des études ont montré que ces classes spéciales constituent des filières dont les élèves s'échappent rarement, bien que l'intégration soit souvent le mot d'ordre de ces dispositifs. Ce choix politique [est donc] de scolariser séparément certains élèves qui ne sont pas et ne seront plus jamais \emph{à l'heure\footnote{Par le terme \emph{à l'heure}, Monceau évoque la progression d'un élève par rapport aux attentes de l'institution en fin d'année de son niveau de classe.}}.} \citep{monceau_groupe_2005}
\\
%En dehors des réflexions de Monceau, nous n'avons pas trouvé d'autre contribution théorique à la définition de la notion de groupe classe.
Ainsi nous n'avons pas trouvé de contribution théorique définissant explicitement la notion de groupe classe.


%\idee{Nous postulons que l'expérimentation d'un vécu pédagogique partagé est bénéfique pour l'apprentissage des élèves.}
\paragraph{Un vécu partagé est bénéfique}
%
% \info{reformulation : ok}
%
Il n'a pas été possible pour nous d'analyser formellement les bienfaits apportés par un vécu partagé au sein du groupe classe. D'abord, comme nous venons de le voir, à cause du manque de précision de la notion de \emph{groupe classe}. Mais aussi face au manque de ressources traitant des bienfaits induits par l'expérience d'un \emph{vécu partagé}.
%
Même si, dans le but de faciliter la gestion de groupe, \cite{peeters_contribution_2018} donne des pistes pour améliorer \textit{le bon esprit ou [la] cohésion de groupe}, il n'analyse pas dans son article les bénéfices ou les bienfaits que l'enseignant peut en tirer pour l'apprentissage des élèves.
%
Toutefois, inspirés par les effets positifs du socio-constructivisme, nous sommes convaincus qu'il est bénéfique pour l'apprentissage des élèves de créer puis de leur proposer des situations permettant d'avoir en commun un vécu pédagogique partagé.



%\idee{Analyser quantitativement l'émergence d'un vécu partagé nous semble difficile puisque celle-ci apparaît dans tous les cas : qu'elle soit naturelle ou forcée par l'enseignant.}
\paragraph{Quantification du vécu partagé}
%
% \info{reformulation : ok}
%
Face à ce postulat, et malgré l'absence de preuve de son intérêt pour l'apprentissage, nous avons commencé à établir une mesure de la connaissance que les élèves ont de leurs camarades. Nous leurs avons proposé un premier questionnaire en début d'année. Notre objectif était de les interroger ainsi à intervalle régulier avec l'idée d'analyser l'évolutions des différents résultats en prenant pour variable le type d'activités proposées selon qu'il favorise ou non l'émergence d'un vécu partagé.
%
Mais nous nous sommes rendus compte que cette démarche ne peut pas prouver que notre dispositif conduit à l'émergence d'un vécu partagé. En effet, le type d'activité ne constitue pas à lui seul la seule variable permettant aux élèves de se connaître
et l'intensité de son influence reste bien minime face à tous les échanges ayant lieu en dehors de ce moment. Il est ainsi évident que la connaissance des élèves entre eux s'améliore nécessairement au cours de l'année et ce, que l'on intervienne ou non dans le processus de connaissance. Notre questionnaire n'aurait pas pu rendre compte de l'importance du type d'activités sur ce phénomène.









\subsection{Émergence de notre dispositif de pédagogie différenciée}

%\idee{Notre recherche s'oriente vers la mise en place d'un dispositif opérationnel concilliant toutes les réflexions précédentes.}
\paragraph{Synthèse des réflexions}
%
% \info{reformulation : ok}
%
L'ensemble de nos recherches et de nos réflexions nous mène à faire évoluer notre questionnement. Nous cherchons donc à mettre en œuvre un dispositif opérationnel de différenciation engendrant le moins possible d'effets pervers dus à la nécessaire individualisation, elle-même accentuée par le morcellement des classes en groupes de spécialité.



%\idee{Une partition des dispositifs différenciés en quatre catégories : le plan de travail, la table d'appui, les groupes de besoins et le tutorat entre élèves.}
\paragraph{Catégorisation des dispositifs de pédagogie différenciée}
%
% \info{reformulation : ok}
%
La recherche de notre dispositif différencié a été facilitée par l'organisation des nombreuses mises en œuvres possibles en quatre catégories. Face aux \textcite{réponses concrètes [qui] demeurent quant à elle plurielles}, \cite{forget_penser_2018} propose de classer les dispositifs de pédagogie différenciée en quatre catégories :
% Ces dernières peuvent être complémentaires et peuvent se concrétiser pendant les trois moments qui sont \textit{avant, pendant et après l'enseignement} :

\begin{description}
	\item [Le plan de travail] qui consiste à construire des progressions individuelles en laissant chaque élève accomplir les tâches d'enseignement à leurs rythmes. Cette catégorie promeut l'\textit{autogestion} de l'élève dans son apprentissage.
	\item [La table d'appui] qui est un lieu au sein de la classe où l'élève peut obtenir une aide à la réalisation de la tâche d'enseignement.
	\item [Les groupes de besoin] qui proposent à tous les élèves (et pas seulement à ceux qui sont le plus en difficulté), des tâches adaptées aux besoins du moment vis-à-vis de l'avancement de leur apprentissage.
	\item [Le tutorat entre élèves] qui propose aux meilleures élèves d'aider les élèves les plus en difficulté sous forme de tutorat en couple.
\end{description}



% La recherche de ce dispositif s'est concrétisée par la lecture d'ouvrages proposant différentes formes de mise en pratique de la pédagogie différenciée. Parmi celles-ci, la synthèse d'nous a paru la plus intéressante à creuser. Dans cet ouvrage, Forget, après avoir rappelé que \textit{les auteurs s'accordent sur les définitions générales de la différenciation pédagogique} mais que \citep[p.18]{forget_penser_2018}, propose de partitionner les dispositifs de pédagogie différenciée en quatre catégories, qui peuvent être complémentaires, à concrétiser pendant les trois temps de l'enseignement, à savoir \textit{avant, pendant et après l'enseignement} (Ibid, p.45) :
% \begin{itemize}
% 	\item Le plan de travail, qui consiste à construire des progressions individuelles en laissant chaque élève accomplir les tâches d'enseignement à leurs rythmes. Cette catégorie promeut l'\textit{autogestion} de l'élève dans son apprentissage.
% 	\item La table d'appui, qui consiste à proposer un lieu au sein de la classe où l'élève peut obtenir une aide à la réalisation de la tâche d'enseignement.
% 	\item Les groupes de besoin, proposant à tous les élèves (et pas seulement à ceux qui sont le plus en difficulté), des tâches adaptées aux besoins du moment vis-à-vis de l'avancement de leur apprentissage.
% 	\item Le tutorat entre élèves, proposant aux meilleures élèves d'aider les élèves les plus en difficulté sous forme de tutorat en couple.
% \end{itemize}






%\idee{Parmi ces catégories, nous choisissons de nous orienter vers le tutorat entre élèves}
\paragraph{Tutorat entre élèves}
%
% \info{reformulation : ok}
%
Après avoir analysé le partitionnement proposé par \cite{forget_penser_2018}, il nous est rapidement devenu évident que le dispositif différencié correspondant le mieux à nos besoins est celui du \emph{tutorat entre élèves}.
\\
Il permet de répondre au problème d'individualisation tout en proposant un cadre socio-constructiviste fort. Toutefois, sensibilisé par les méfaits de la stigmatisation, une mise en œuvre naïve qui définirait quelques élèves \emph{tuteurs} et les autres \emph{apprenants} nous semble contre productive. C'est pourquoi le point central de notre dispositif est de permettre à tous les élèves, y compris ceux le plus en difficulté, d'être tantôt l'apprenant et tantôt le tuteur source du savoir pour ses camarades. La transmission du savoir entres les élèves se faisant dans les deux sens, notre dispositif sera donc un tutorat \emph{bidirectionnel} entre élèves.

Il nous faut maintenant faire face à de nombreuses difficultés. Comment former des groupes d'élèves ayant des zones proximales de développement sécantes ? Est-il possible de rendre tous les élèves tuteurs ? 


% L'analyse de ces différentes catégories nous a amenés à nous questionner sur celle que nous allions privilégier. Le tutorat entre élèves est très vite apparu comme évident à nos yeux, car il permet de répondre au problème d'individualisation en forçant l'échange entre élèves, et surtout de mettre en pratique le socio-constructivisme de Vygotsky, la plus grosse difficulté étant de former des couples d'élèves ayant une \textit{zone proximale de développement} similaire. Ces quatre catégories comportaient toutes un gros risque de stigmatisation entre les \og bons \fg{} et les \og moins bons \fg{}. Il nous a semblé être plus facile de contourner ce problème grâce au tutorat entre élèves, car nous savions que nous pourrions trouver, même chez les élèves les plus en difficulté, un point sur lequel ils pourraient apporter aux élèves en ayant moins.

\vspace{1cm}

{\Huge FIN ?}

\subsection{Pédagogie de tutorat bidirectionnelle}
%\idee{Nous nous proposons donc d'utiliser la notion de pédagogie différenciée pour construire une pédagogie collaborative permettant à tous les élèves d'apporter chacun leurs compétences.}
\paragraph{Pédagogie différenciée collaborative}
Partant des constats et réflexions menés jusqu'à présent, nous avons réfléchi à une méthode de tutorat entre élèves qui serait collaborative, mais surtout qui inclurait tous les élèves quels que soient leurs niveaux. Nous voulions éviter de stigmatiser certains élèves en proposant un tutorat unidirectionnel où les \og meilleurs \fg{} aident les \og moins bons \fg{}. C'est en commençant à construire cette méthode que nous nous sommes aperçus qu'elle ressemblait grandement à la méthode Jigsaw.

%\idee{Jigsaw est apparu dans les années 1970 au USA, en pleine période de déségrégation raciale.}
\paragraph{Histoire de Jigsaw}
Cette technique d'enseignement a été créée dans les années 1970 par Elliot Aronson, actuellement Professeur à l'Université de Californie à Santa Cruz. À cette époque, la déségrégation venait de débuter dans les écoles américaines, et les jeunes américains issus des communautés blanches, noires et hispaniques se sont retrouvés pour la première fois ensemble en classe. Après quelques semaines, la peur et le manque de confiance entre les groupes d'élèves commençaient à créer des tensions. Il fallait donc trouver une solution pour faire en sorte que les groupes d'élèves s'entendent entre eux.


%\idee{Jigsaw est une technique d'enseignement coopérative qui promeut l'écoute, l'engagement, l'interaction et le partage.}
\paragraph{Présentation de la méthode Jigsaw}
Aronson a donc imaginé et mis en place la méthode Jigsaw : le but était que pour chaque notion du programme, celle-ci soit découpée en tâches essentielles à la compréhension de la notion, et que chaque tâche soit étudiée par un élève issu d'une communauté différente. Ainsi, chaque élève \og expert \fg{} de sa tâche est essentiel à la compréhension de la notion par tout le groupe. Cette technique a ensuite été généralisée pour ne plus seulement travailler la déségrégation, mais plus globalement pour permettre à chaque élève d'apporter quelque chose au groupe, pour ainsi se sentir moins seul et moins exclu. La mise en place d'une telle technique permet également aux élèves de développer des compétences humaines telles que l'écoute et le partage, car chaque élève doit apprendre et enseigner à l'autre pour acquérir la notion dans sa globalité.

Concrètement, Jigsaw se décompose en 10 étapes :
\begin{enumerate}
	\item Créer des groupes Jigsaw de 5-6 étudiants de niveaux, genres et origines hétérogènes.
	\item Nommer un leader pour chaque groupe, qui sera chargé de médiation au sein du groupe.
	\item Diviser la leçon du jour en autant de sous-leçons qu'il y a d'étudiants au sein de chaque groupe.
	\item Assigner une sous-leçon à chaque membre du groupe.
	\item Laisser à chaque étudiant le temps de se familiariser avec sa sous-leçon (au moins 2 lectures du thème).\label{Familiarisation}
	\item Former des \emph{groupes d'experts} en regroupant les étudiants ayant la même sous-leçon à étudier. Laisser à chaque groupe d'experts le temps d'échanger sur la sous-leçon afin de pouvoir tous présenter le même contenu aux groupes Jigsaw.
	\item Reformer les groupes Jigsaw.
	\item Demander à chaque étudiant de présenter la sous-leçon dont il est expert à son groupe.\label{PrésentationSousLecon}
	\item Naviguer entre les groupes pour observer et éventuellement intervenir en cas de problème.
	\item Évaluer les étudiants sur la leçon dans sa globalité.
\end{enumerate}

%\idee{Notre méthode de tutorat bidirectionnel est proche de celle de Jigsaw}
\paragraph{Notre Jigsaw}
Instinctivement, nous avions imaginé une méthode extrêmement proche de Jigsaw, et le fait que Jigsaw existe nous a conforté dans notre idée qu'elle pourrait permettre de répondre à nos attentes. Notre méthode, présentée sur la figure \ref{SchémaNotreJigsaw}, diffère de Jigsaw sur plusieurs points :
\begin{itemize}
	\item Pour des questions d'effectifs et de décomposition des leçons que nous avons choisies, nos groupes Jigsaw ne sont composés que de 3 à 5 élèves.
	\item L'hétérogénéité de nos groupes Jigsaw n'est basée que sur les niveaux des élèves, ayant très peu de filles et n'ayant pas suffisamment d'origines ethniques différentes.
	\item Nous ne nommons pas de leader dans les groupes Jigsaw, voulant appuyer le côté collaboratif et ne voulant pas valoriser certains élèves au détriment d'autres.
	\item Nous n'effectuons pas l'étape \ref{Familiarisation}, par manque de temps pour chaque séance et considérant que le côté familiarisation est effectué à l'étape suivante de manière collective.
	\item L'étape \ref{PrésentationSousLecon} consiste en la résolution par le groupe Jigsaw d'un exercice pour chaque notion issue de chaque sous-leçon.
\end{itemize}


\newpage

\newgeometry{margin=2cm,left=5cm,marginparwidth=4cm}
\fancyhfoffset[L]{3cm}



\subsection{Idées non classées pour le moment mais qui peuvent s'avérer intéressantes}


\idee{Dans la pédagogie de tutorat, l'échange de savoirs entre élèves peut être vu comme la genèse instrumentale du savoir.} 
Le cadre théorique dans lequel nous nous inscrivons est celui de \cite{rabardel_les_1995} lorsqu'il distingue artefact et instrument. L'artefact est l'objet, l'outil ou dans notre cas la ressource mise à disposition par l'enseignant auprès du groupe d'appropriation. L'instrument est une entité composée de l'artefact, ou d'une partie de celui-ci, et la mise en action de ce dernier par l'élève, l'adaptation qu'il en fait pour l'accommoder à ses besoins lorsqu'il doit le transmettre à ses camarades.


\idee{La pédagogie différenciée contemporaine}
De nos jours, de nombreux travaux s'appuient sur ceux de Piaget et Freinet, pour proposer des solutions pratiques à la pédagogie différenciée. Bien différentes les unes des autres, ces études nous montrent bien qu'il n'y a pas de méthode miracle, mais que le plus important pour un enseignant est de se questionner quotidiennement sur l'efficacité des méthodes qu'il emploie et de s'adapter aux différentes situations de classe.





\idee{Kahn expose 3 grandes questions sur la différenciation}
Cette difficulté de pratique de la différenciation se retrouve chez \cite{kahn_pedagogie_2010}, pour qui la définition même de la pédagogie différenciée peut prêter à discussions. Elle expose trois grandes questions sur les conceptions contemporaines de la différenciation : 
\begin{itemize}
	\item \textcite{La pédagogie différenciée [\ldots] touche à la notion de différences entre les individus. [\ldots] L'école est aux prises avec des injonctions divergentes qui vont de la sacralisation de la différence à l'effort vers la construction d'une universalité.}
	\item \textcite{\emph{Se préoccuper} des différences entre élèves peut s'entendre en deux sens opposés : on peut vouloir les sauvegarder ou on peut vouloir les réduire.}
	\item \textcite{L'idée de \emph{traiter} les différences entre élèves [\ldots] peut se faire au niveau de l'institution elle-même par la diversification du cursus en filières différentes, [\ldots] entre les classes d'une même filière à l'échelle d'un établissement, [\ldots] au niveau des élèves d'une même classe en prévoyant pour certains d'entre eux des dispositifs de remédiation, [mais] aussi au sein même de la classe au moyen d'activités différentes selon les élèves.}
\end{itemize}



\idee{Cheminement de notre réflexion}
Parler de notre réflexion initiale sur l'appartenance au groupe classe, et la difficulté de mise en œuvre d'une pédagogie différenciée dans une classe avec des élèves qui ne se connaissent pas. Et notre bifurcation vers la pédagogie collaborative, que l'on pense être une forme de pédagogie différenciée qui peut répondre au problème de non-connaissance des élèves entre eux.

\newgeometry{margin=2cm}
\fancyhfoffset[L]{0cm}