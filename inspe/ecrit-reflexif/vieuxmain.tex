\documentclass[a4paper]{article}

\usepackage[utf8]{inputenc}
\usepackage[OT1]{fontenc}
\usepackage[french]{babel}
\usepackage{geometry}
\usepackage{hyperref}

\title{Notre titre général}
\author{François Decq -- Pascal Padilla}
\date{Année académique 2020-2021}

\begin{document}
\maketitle\newpage

\section*{Introduction}
Idée de base : pédagogie différenciée en NSI.
\begin{itemize}
	\item Les élèves ne se connaissent pas (dû à la création des enseignements de spécialité comme c'est le cas pour NSI) $\rightarrow$ Dans un premier temps, questionnaire évaluant la connaissance des élèves entre eux.
	\item Contexte de la NSI :
	\begin{itemize}
		\item Nouvelle discipline.
		\item Mauvais a priori de la société (vision faussée de l'informatique) donc mauvais a priori des élèves.
		\item Matière technique et théorique.
		\item Niveaux initiaux des élèves très hétérogènes entre ceux qui découvrent l'informatique et ceux qui ont déjà quelques notions.
		\item Ceux qui ont déjà des notions ne sont pas forcément meilleurs, au contraire, car ils ont construit des connaissances à partir d'une expérience d'utilisateur plutôt que d'une expérience de concepteur.
	\end{itemize}
	\item Pédagogie différenciée : à définir à partir de références institutionnelles et théoriques et de lectures de mémoires.
	\item Liens et différences entre pédagogie différenciée et hétérogénéité.
	\item Réflexions autour de l'émergence de la notion d'\og Amélioration des apprentissages \fg{} comme sujet principal. Qu'est-ce qu'améliorer les apprentissages ?
\end{itemize}



En tant qu'enseignants de l'Éducation Nationale, nous avons le rôle de former des citoyen.ne.s en représentant auprès des usagers de l'École un certain nombre de principes qui peuvent être résumés aux \og Valeurs de la République \fg{}. Une de valeurs les plus fondamentales est de pouvoir garantir à toutes et à tous les élèves d'accéder aux savoirs, de les accueillir quels que soient leurs origines et leurs parcours.

Ainsi, en classe, chaque élève arrive avec son parcours. L'objectif pour l'enseignant est de l'accueillir pour lui partager son savoir tout en répondant à ses attentes et ses sollicitations individuelles. Un élève n'étant jamais seul en classe, l'enseignant doit donc pouvoir répondre aux attentes et sollicitations de l'ensemble de ses élèves. La mise en œuvre de telle problématique a été théorisée au travers ce que l'on appelle la pédagogie différenciée. L'objectif primordiale de l'école devient donc
\begin{quote} de donner à tous des chances d'apprendre, quelles que soient son origine sociale et ses ressources culturelles. \cite{Perrenoud}\end{quote}\newpage

\section{Cadre théorique}
Peut-être partir des références institutionnelles comme les programmes scolaires ou Eduscol qui prônent, via les \og valeurs de la République \fg{} comme la liberté, l'égalité, la fraternité ou la laïcité, l'inclusion de tous (inclusion au sens large, pas uniquement le thème du handicap) : différences physiques et psychologiques.

À partir de ces textes, pointer le fait qu'il y a la théorie et la pratique, que tout ça est difficile à mettre en œuvre au quotidien car les différences existent tout le temps et pour tout le monde, que ce soit au niveau individuel (différences de niveau, de compréhension) comme au niveau collectif (les classes sont toutes différentes entre elles).

Ainsi, au niveau théorique, beaucoup de recherches sur la pédagogie différenciée (et c'est là qu'on peut citer les différentes références biblio qu'on a) et terminer en pointant le fait qu'il est difficile de passer de la théorie à la pratique, et que l'on propose une forme de pédagogie différenciée qui ne prétend pas résoudre tous les problèmes.\newpage

\section{Questions de recherche}
\begin{itemize}
	\item Un élève qui transmet une notion à ses camarades améliore ses apprentissages sur cette notion.
	\item Un élève qui reçoit une notion transmise par un camarade voit son apprentissage de la notion amélioré.
\end{itemize}
À différencier de la notion de tutorat, pour lequel la transmission est unidirectionnelle. Ici la transmission est bidirectionnelle (les \og bons \fg{} transmettent aux \og moins bons \fg{} mais les \og moins bons \fg{} transmettent aussi aux \og bons \fg{}).\newpage

\section{Méthodologie de recherche}
Premières NSI Monte-Cristo Allauch et Simone Veil Marseille 13\newpage

\section{Résultats et analyses des données recueillies}
\newpage

\section{Discussions}
QCM $\rightarrow$ Apprendre\\
Comprendre ?\\
Avec le QCM, a-t-on bien évalué la connaissance de l'élève ?\\


Collaboration $\rightarrow$ Sur le long terme les élèves gagnent en confiance en eux car ils se sentent inclus au sein du groupe (sans jugement des autres ?)\\


Évaluations avant séquence et un mois après $\rightarrow$ Diagnostic des connaissances mais pas de compétences ?\\\newpage


\section*{Références biblio}
\begin{itemize}
	\item \cite{Lautru} : lecture intéressante sur laquelle se baser pour la partie théorique, peut-être moins sur la partie pratique.
	\item \cite{Meirieu} : apports théoriques difficiles à appliquer, mais certaines idées et citations peuvent être intéressantes à reprendre pour compléter la partie théorique.
	\item \cite{Forget} : différentes formes de différenciation pédagogique, dont le tutorat entre élèves.
\end{itemize}

\begin{thebibliography}{4}
\bibitem{Lautru} 
\bibitem{Meirieu}
\bibitem{Forget}
\bibitem{Perrenoud} Perrenoud, P. (1997). Pédagogie différenciée : des intentions à l’action. Paris, France : ESF éditeur.
\end{thebibliography}
\end{document}