À partir des réflexions menées pour affiner les questions que nous nous posions, la question que nous avons retenue est la suivante :

\textbf{Une pédagogie différenciée de tutorat entre élèves bidirectionnel permet-elle d'améliorer l'apprentissage en enseignement de spécialité ?}

De cette question découlent 2 hypothèses :
\begin{itemize}
	\item Un élève qui transmet une notion à ses camarades améliore ses apprentissages sur cette notion.
	\item Un élève qui reçoit une notion transmise par un camarade voit son apprentissage de la notion amélioré.
\end{itemize}

Cette forme pédagogie différenciée est à différencier de la notion de tutorat, pour lequel la transmission est unidirectionnelle. Ici la transmission est bidirectionnelle (les \og bons \fg{} transmettent aux \og moins bons \fg{} mais les \og moins bons \fg{} transmettent aussi aux \og bons \fg{}).